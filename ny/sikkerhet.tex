Dette kapittelet tar for seg sikkerhetstiltakene og reglene som gruppen har fulgt. Målene for gruppen var å: Ikke skade personer eller omgivelsene med dronen. Ikke ødelegge kritiske komponenter. Ikke interferere andre sendinger i nærheten, f.eks flytrafikk og radiosamband.

\subsection{Sikkerhetstiltak for operasjonene}
Det ble utført flere tiltak for å få dronen trygg for innendørs bruk. Gruppen vurderte at flyvninger med udekket propeller kunne få for store konsekvenser.  Det ble derfor designet propellerbeskyttere til dronen.
Gruppen har bare 2 UWB-tags tilgjengelig, og disse må ikke bli ødelagt. For å unngå å skade disse ble det designet ett “rullebur” for å forhindre skade under krasj eller harde landinger. 
De 3d-printet delene ble designet for å deformere seg eller knekke for å minke nedslag til dronen sine kritiske komponenter. Dette ble valgt siden 3d-printet deler er enkelt og billig å bytte ut. 

\subsection{Tiltak for drone på avveie}
For å alltid ha kontroll over dronen ble det brukt enn manuell-pilot under flyvningene. Den manuell-piloten sin oppgave var å ta over kontroll om drone kom på avveie på grunn av programfeil osv.  Det ble også brukt basestasjon for å kontrollere og overvåke dronen. Personell på basestasjon overvåkte kritiske verdier som batteribruk og helsen til sensorene. 
Ved tap av manuell-kontroll link ble dronen programmert til å kutte motorene, dette ble valgt siden dronen kan uføre større skade med å prøve å lande ta seg selv 
For å unngå brukerfeil under flyvning ble sjekklister implementert. 

\subsection{Regelverk for droneflyvning}
For innendørsflyvning gjelder ikke regelverket for droner. Slik at her kunne en flyve uten operasjonslisens. 
For utendørsflyvning har en av gruppens deltaker registrert seg i åpen kategori og har kommersiell forsikring. 

\subsection{Regelverk for sendinger}
For sendingen i prosjektet er det ingen krav for tillatelser for bruk, men det er regler for hvilke frekvenser som kan brukes, hvilken sendestyrke som er lovlig og hvordan data blir sendt.
\begin{itemize}
\item Regelverket for kommunikasjon til dronen gjelder under forskrift om generelle tilatelser til bruk av frekvener (fribruksfirskriften), kap 3.8 Diverse utstyr for kortdistansekommunikasjon. https://lovdata.no/forskrift/2012-01-19-77/§33.8   
\item For regelverket for UWB posisjon blir regelverket for generelle tilatelser til bruk av frekvener (fribruksfirskriften), kap 35a følgt.   https://lovdata.no/forskrift/2012-01-19-77/§35a 
\end{itemize}